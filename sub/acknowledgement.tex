\documentclass[../main]{subfiles}
\begin{document}

\begin{acknowledgement}

  感谢康老师的悉心指导!

  参考文献是霍夫曼的论文,感谢!
  
  代码没有参考任何人。Julia 关于编码的函数包
  \href{https://github.com/oschulz/EncodedArrays.jl}{oschulz/EncodedArrays.jl}
  2020年5月才由\href{https://github.com/oschulz}{Oliver Schulz}着手开发,只提
  供了编解码的抽象类声明的构造函数,具体编解码算法实现的具象类恐怕还要等一段
  时间。唯一与霍夫曼编码和 Julia 都相关的是一篇 IEEE 的
  \href{https://ieeexplore.ieee.org/document/7754207}{论文},但我没有办法查看
  ,猜测他应该和我一样,没有用面向对象封装。但看摘要,他们开发程序的目的是为
  了提高学生对编码技术的理解,所以实现了代码的可视化,会自动画出霍夫曼二叉树。
  如果您乐意,我也可以利用 Vim script 实现一个可视化界面。

  本文开源于
  \href{https://github.com/Freed-Wu/encode-report}{Freed-Wu/encode-report},
  欢迎留言指正!
\end{acknowledgement}

\end{document}

